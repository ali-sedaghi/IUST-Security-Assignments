\Problem
{زمان حمله \lr{Brute-force}}
{
در این نوع حملات کل فضای حالت بایستی پیمایش شود. البته در صورتی که پاسخ رسیدیم ادامه فضا را بررسی نمی‌کنیم. اعدادی که در ادامه داده خواهد حالت
\lr{Worst case}
می‌باشد.
بدیهی است هرچه طول کلید بیشتر باشد فضای حالت نیز بزگ‌تر می‌باشد و این نوع حمله زمان بیشتری برای رمزگشایی نیاز دارد.
در این نوع حملات همه حالات ممکن برای کلید یکی یکی بررسی می‌شوند پس طبیعی است که زمان زیادی طول بکشد.
در گذشته این عملیات روی
\lr{CPU}
صورت می‌گرفت که تعداد هسته کمی داشتند و قدرت
\lr{TFLOPS}
آن‌ها پایین بود.
امروزه با معرفی کارت‌های گرافیک
\lr{(GPU)}
قدرتمند که دارای هسته‌های زیادی هستند و سرعت عملیات اعشاری در آن‌ها زیاد است، این زمان کمتر شده است.
سه فاکتور مهم در مدت زمان این حملات شامل الگوریتم، طول کلید و سخت افزار است.

اگر فرض کنیم در کلید از تمامی کاراکترهای موجود در
\lr{ASCII}
استفاده شده باشد پس 95 کاراکتر داریم.
عامل تاثیرگذار دیگر طول رمز می‌باشد.
اگر یک
\lr{CPU}
هشت هسته‌ای بخواهد همچین رمزی را بشکند نیازمند زمانی زیاد است. زیرا این سخت افزار تقریبا در هر ثانیه حدود نیم میلیون رمز را امتحان می‌کند، اما فضای حالت مسئله برابر 8 به توان 95 است.

اگر از یک
\lr{GPU}
قدرتمند مانند
\lr{NVIDIA RTX 3090}
که جزو کارت‌های گرافیک جدید این شرکت است استفاده کنیم زمان بسیار کمتری طول خواهد کشید.
این سخت افزار می‌تواند تا چند میلیون رمز را در ثانیه امتحان کند.
چیزی حدود 27 میلیون رمز در ثانیه
با حساب کتابی ساده می‌توان گفت این سخت افزار نیازمند 68200 ساعت برای شکاندن رمز است.
این عدد معادل 8 سال زمان است!

یکی از کارهایی که برای کاهش این زمان انجام می‌دهند موازی سازی است. به این ترتیب که از چند کامپیوتر به طور همزمان برای شکاندن یک رمز استفاده می‌کنند.
همچنین کامپیوترهای کوانتمی نیز جهش بزرگی در محسابات دارند و قدرت حدس 63 میلیارد رمز در ثانیه را دارند و می‌توانند رمزها را در زمان کوتاه‌تری بگشایند. برای مثال این سیستم‌ها قادر هستند رمز 8 کاراکتری را در 30 ساعت رمزگشایی کنند. همچنین این عدد برای رمز 10 کاراکتری به 30 سال می‌رسد.

اگر یک سیستم غیرکوانتومی را در نظر بگیریم رمز 10 کاراکتری حدود 525 سال زمان می‌برد تا شکسته شود. این زمان برای تعداد کاراکتر پایین بسیار کوتا هست مثلا رمز 5 کاراکتری در کسری از ثانیه شکسته می‌شود. اما رمز 6 کاراکتری حدود 10 ثانیه زمان می‌برد.
}
