\Problem
{انواع رمز‌های جانشینی}
{
همانطور که در کلاس درس نیز بررسی شد، در این نوع رمزنگاری به جای یک یا تعدادی از کاراکترها یک یا چند کاراکتر دیگر قرار می‌گیرد. در ادامه چند روش از این نوع رمزنگاری را بررسی می‌کنیم:

\begin{itemize}
    \item \lr{Simple Substitution Cipher}:
    رایج ترین روش است. به ازای هر حرف یک حرف دیگر قرار می‌گیرد. فضای حالت آن معادل 26 فاکتوریل است. این نوع رمزها در برابر تحلیل فرکانسی بسیار شکننده هستند. اگر ترتیب جایگذاری حروف را تعریف کنیم می‌توانیم به رمزنگاری سزار برسیم.
    
    \item \lr{Poly-gram Substitution Cipher}:
    در این روش بلاک‌هایی از کاراکترها با بلاکی دیگر جابجا می‌شود. یکی از نمونه این نوع جانشینی الگوریتم
    \lr{Hill Cipher}
    است.
    این نوع رمزنگاری در برابر تحلیل فرکانسی مقاوم است. یک نکته همه دیگر این است که تشابه حروف دو بلاک به معنی تشابه حروف رمز شده نیست.
    این نوع رمزها در دوران رنسانس بسیار مورد استقبال بودند.
    
    \item \lr{Homo-phonic Substitution Cipher}:
    در این حالت یک کاراکتر می‌تواند به چند کاراکتر نگاشت پیدا کند. مثلا می‌توانیم یک حرف را با چهار حرف جایگزین کنیم.
    نمونه معروف این نوع رمزنگاری الگوریتم‌های
    \lr{Beale}
    است.
    
    \item \lr{Poly-alphabetic Substitution Cipher}:
    در این روش قانون و نگاشت جانشینی ثابت نیست و در طول رشته می‌تواند تغییر کند. مثلا کاراکتر
    \lr{A}
    می‌تواند در ابتدا به
    \lr{P}
    نگاشت شود.
    اما در تکرار دوم
    \lr{A}
    به
    \lr{M}
    نگاشت شود.
    رمزنگاری ویگنر و انیگما از این نوع است.
    این نوع روش‌ها برای اولین بار توسط
    \lr{Leon Battista}
    معرفی شد.
\end{itemize}

برخی از روش‌های بالا را می‌توان با هم ترکیب کرد و موارد جدید ساخت.
مثلا
\lr{Poly-alphabetic and Homo-phonic}
}
