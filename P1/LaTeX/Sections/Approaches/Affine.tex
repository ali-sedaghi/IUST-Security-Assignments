\SubProblem
{مستوی}
{
رمز سزار بسیار ساده بود و به سادگی قابل شکسته شدن بود.
بنابراین روش کلی‌تر آن یعنی رمز مستوی
\lr{(Affine Cipher)}
ارائه شد.

رابطه کلید در آن به صورت زیر است:
\begin{equation*}
    c = E_k(m) = (a \times m + k) \mod 26
\end{equation*}

تابع رمزگشایی:
\begin{equation*}
    p = D_k(m) = \frac{(26 \times m - b)}{a}
\end{equation*}

در این روش با دو متغیر
\lr{a} و \lr{k}
در طرف هستیم پس قدرت الگوریتم بیشتر از سزار است.
نوع کلی‌تر این الگوریتم نیز موجود است که طول کلید در آن 88 بیت است!

همانطور که در کلاس درس استاد مطرح کردند، این متن توسط این الگوریتم رمز شده است و تنها پیاده‌سازی این روش برای رمزگشایی کافی است.
}
